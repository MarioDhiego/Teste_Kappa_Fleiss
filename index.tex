% Options for packages loaded elsewhere
\PassOptionsToPackage{unicode}{hyperref}
\PassOptionsToPackage{hyphens}{url}
\PassOptionsToPackage{dvipsnames,svgnames,x11names}{xcolor}
%
\documentclass[
  letterpaper,
  DIV=11,
  numbers=noendperiod]{scrreprt}

\usepackage{amsmath,amssymb}
\usepackage{lmodern}
\usepackage{iftex}
\ifPDFTeX
  \usepackage[T1]{fontenc}
  \usepackage[utf8]{inputenc}
  \usepackage{textcomp} % provide euro and other symbols
\else % if luatex or xetex
  \usepackage{unicode-math}
  \defaultfontfeatures{Scale=MatchLowercase}
  \defaultfontfeatures[\rmfamily]{Ligatures=TeX,Scale=1}
\fi
% Use upquote if available, for straight quotes in verbatim environments
\IfFileExists{upquote.sty}{\usepackage{upquote}}{}
\IfFileExists{microtype.sty}{% use microtype if available
  \usepackage[]{microtype}
  \UseMicrotypeSet[protrusion]{basicmath} % disable protrusion for tt fonts
}{}
\makeatletter
\@ifundefined{KOMAClassName}{% if non-KOMA class
  \IfFileExists{parskip.sty}{%
    \usepackage{parskip}
  }{% else
    \setlength{\parindent}{0pt}
    \setlength{\parskip}{6pt plus 2pt minus 1pt}}
}{% if KOMA class
  \KOMAoptions{parskip=half}}
\makeatother
\usepackage{xcolor}
\setlength{\emergencystretch}{3em} % prevent overfull lines
\setcounter{secnumdepth}{5}
% Make \paragraph and \subparagraph free-standing
\ifx\paragraph\undefined\else
  \let\oldparagraph\paragraph
  \renewcommand{\paragraph}[1]{\oldparagraph{#1}\mbox{}}
\fi
\ifx\subparagraph\undefined\else
  \let\oldsubparagraph\subparagraph
  \renewcommand{\subparagraph}[1]{\oldsubparagraph{#1}\mbox{}}
\fi


\providecommand{\tightlist}{%
  \setlength{\itemsep}{0pt}\setlength{\parskip}{0pt}}\usepackage{longtable,booktabs,array}
\usepackage{calc} % for calculating minipage widths
% Correct order of tables after \paragraph or \subparagraph
\usepackage{etoolbox}
\makeatletter
\patchcmd\longtable{\par}{\if@noskipsec\mbox{}\fi\par}{}{}
\makeatother
% Allow footnotes in longtable head/foot
\IfFileExists{footnotehyper.sty}{\usepackage{footnotehyper}}{\usepackage{footnote}}
\makesavenoteenv{longtable}
\usepackage{graphicx}
\makeatletter
\def\maxwidth{\ifdim\Gin@nat@width>\linewidth\linewidth\else\Gin@nat@width\fi}
\def\maxheight{\ifdim\Gin@nat@height>\textheight\textheight\else\Gin@nat@height\fi}
\makeatother
% Scale images if necessary, so that they will not overflow the page
% margins by default, and it is still possible to overwrite the defaults
% using explicit options in \includegraphics[width, height, ...]{}
\setkeys{Gin}{width=\maxwidth,height=\maxheight,keepaspectratio}
% Set default figure placement to htbp
\makeatletter
\def\fps@figure{htbp}
\makeatother

\KOMAoption{captions}{tableheading}
\makeatletter
\makeatother
\makeatletter
\@ifpackageloaded{bookmark}{}{\usepackage{bookmark}}
\makeatother
\makeatletter
\@ifpackageloaded{caption}{}{\usepackage{caption}}
\AtBeginDocument{%
\ifdefined\contentsname
  \renewcommand*\contentsname{Table of contents}
\else
  \newcommand\contentsname{Table of contents}
\fi
\ifdefined\listfigurename
  \renewcommand*\listfigurename{List of Figures}
\else
  \newcommand\listfigurename{List of Figures}
\fi
\ifdefined\listtablename
  \renewcommand*\listtablename{List of Tables}
\else
  \newcommand\listtablename{List of Tables}
\fi
\ifdefined\figurename
  \renewcommand*\figurename{Figure}
\else
  \newcommand\figurename{Figure}
\fi
\ifdefined\tablename
  \renewcommand*\tablename{Table}
\else
  \newcommand\tablename{Table}
\fi
}
\@ifpackageloaded{float}{}{\usepackage{float}}
\floatstyle{ruled}
\@ifundefined{c@chapter}{\newfloat{codelisting}{h}{lop}}{\newfloat{codelisting}{h}{lop}[chapter]}
\floatname{codelisting}{Listing}
\newcommand*\listoflistings{\listof{codelisting}{List of Listings}}
\makeatother
\makeatletter
\@ifpackageloaded{caption}{}{\usepackage{caption}}
\@ifpackageloaded{subcaption}{}{\usepackage{subcaption}}
\makeatother
\makeatletter
\@ifpackageloaded{tcolorbox}{}{\usepackage[many]{tcolorbox}}
\makeatother
\makeatletter
\@ifundefined{shadecolor}{\definecolor{shadecolor}{rgb}{.97, .97, .97}}
\makeatother
\makeatletter
\makeatother
\ifLuaTeX
  \usepackage{selnolig}  % disable illegal ligatures
\fi
\IfFileExists{bookmark.sty}{\usepackage{bookmark}}{\usepackage{hyperref}}
\IfFileExists{xurl.sty}{\usepackage{xurl}}{} % add URL line breaks if available
\urlstyle{same} % disable monospaced font for URLs
\hypersetup{
  pdftitle={Teste Não-Paramétrico Kappa},
  pdfauthor={Mário Diego Rocha Valente},
  colorlinks=true,
  linkcolor={blue},
  filecolor={Maroon},
  citecolor={Blue},
  urlcolor={Blue},
  pdfcreator={LaTeX via pandoc}}

\title{\textbf{Teste Não-Paramétrico Kappa}}
\usepackage{etoolbox}
\makeatletter
\providecommand{\subtitle}[1]{% add subtitle to \maketitle
  \apptocmd{\@title}{\par {\large #1 \par}}{}{}
}
\makeatother
\subtitle{\textbf{Exemplos no R}}
\author{\textbf{Mário Diego Rocha Valente}}
\date{5/2/26}

\begin{document}
\maketitle
\ifdefined\Shaded\renewenvironment{Shaded}{\begin{tcolorbox}[interior hidden, sharp corners, borderline west={3pt}{0pt}{shadecolor}, boxrule=0pt, enhanced, breakable, frame hidden]}{\end{tcolorbox}}\fi

\renewcommand*\contentsname{Table of contents}
{
\hypersetup{linkcolor=}
\setcounter{tocdepth}{2}
\tableofcontents
}
\bookmarksetup{startatroot}

\hypertarget{prefuxe1cio}{%
\chapter*{Prefácio}\label{prefuxe1cio}}
\addcontentsline{toc}{chapter}{Prefácio}

\markboth{Prefácio}{Prefácio}

O coeficiente Kappa, simbolizado pela letra grega minúscula κ e criado
pelo estatístico Jacob Cohen (1960), mede a concordância entre dois
avaliadores, cada um classificando N itens em C categorias mutuamente
exclusivas.

O Kappa de Cohen é o mais usado e avalia a concordância entre dois
examinadores (ou dois métodos). Por outro lado, Fleiss (1981) propôs uma
extensão do Kappa para o caso em que há mais de dois examinadores (ou
métodos), que foi denominada Kappa Generalizado ou Kappa de Fleiss.

Outra extensão do Kappa com grande aplicabilidade é o Kappa Ponderado,
que visa distinguir as discordâncias/concordâncias atribuindo pesos
diferentes para cada tipo (por exemplo, em leves, moderadas e graves).

\bookmarksetup{startatroot}

\hypertarget{teste-nuxe3o-paramuxe9trico-de-kappa}{%
\chapter{Teste Não-Paramétrico de
Kappa}\label{teste-nuxe3o-paramuxe9trico-de-kappa}}

\hypertarget{coeficiente-kappa-de-cohen}{%
\section{coeficiente kappa de Cohen}\label{coeficiente-kappa-de-cohen}}

O coeficiente kappa de Cohen (\(κ\)) é uma estatística que é usada para
medir a confiabilidade interexaminador (e também a confiabilidade
intraexaminador) para itens qualitativos (categóricos). Geralmente é
considerada uma medida mais robusta do que o simples cálculo percentual
de concordância, pois \(κ\) leva em consideração a possibilidade de a
concordância ocorrer por acaso. Há controvérsia em torno do kappa de
Cohen devido à dificuldade em interpretar os índices de concordância.
Alguns pesquisadores sugeriram que é conceitualmente mais simples
avaliar a discordância entre os itens.

O artigo seminal introduzindo o kappa como uma nova técnica foi
publicado por \textbf{Jacob Cohen} na revista Educational and
Psychological Measurement em 1960.

Jacob Cohen (20 de abril de 1923 - 20 de janeiro de 1998) foi um
psicólogo e estatístico norte-americano mais conhecido por seu trabalho
sobre potência estatística e tamanho do efeito, que ajudou a estabelecer
as bases para a meta-análise estatística atual e os métodos da
estimativa estatística. Ele deu seu nome a medidas como kappa de Cohen,
d de Cohen e h de Cohen.

O kappa de Cohen mede a concordância entre dois avaliadores que
classificam N itens em C categorias mutuamente exclusivas. A definição
de \emph{kappa} é:

\[ \kappa = \frac{p_{0}-p_{1}}{1-p_{e}} = 1- \frac{1-p_{0}}{1-p_{e}}\]

Onde:

\begin{itemize}
\tightlist
\item
  \(p_{0}:\) é a concordância relativa observada entre os avaliadores
\item
  \(p_{e}:\) é a probabilidade hipotética de concordância ao acaso,
  usando os dados observados para calcular as probabilidades de cada
  observador ver aleatoriamente cada categoria.
\item
  Se os avaliadores estiverem de acordo, então \(\kappa = 1\).
\item
  Se não houver acordo entre os avaliadores além do que seria esperado
  por acaso (conforme dado por \(p_{e}\)), então \(\kappa = 0\).
\end{itemize}

É possível que a estatística seja negativa, o que pode ocorrer por acaso
se não houver relação entre as avaliações dos dois avaliadores, ou pode
refletir uma tendência real dos avaliadores em dar avaliações
diferentes.

A estatística pode variar de -1 a +1, onde o valor negativo indica que a
concordância entre os avaliadores foi menor que a concordância esperada
ao acaso. Com -1 estamos indicando que não houve concordância, 0 indica
que a concordância não é melhor que o acaso, e valores maiores que 0
representam uma concordância crescente para os avaliadores, até um valor
máximo de + 1, indicando uma concordância perfeita.

\hypertarget{coeficiente-kappa-fleiss}{%
\section{Coeficiente kappa Fleiss}\label{coeficiente-kappa-fleiss}}

O Kappa de Cohen é o mais usado e avalia a concordância entre dois
examinadores (ou dois métodos). Por outro lado, Fleiss (1981) propôs uma
extensão do Kappa para o caso em que há mais de dois examinadores (ou
métodos), que foi denominada Kappa Generalizado ou Kappa de Fleiss.

O Kappa de Fleiss é um teste estatístico não-paramétrico, utilizado para
avaliar o grau de concordância entre 3 ou + observadores, avaliadores
e/ou juízes.

Joseph L. Fleiss (13 de novembro de 1937 - 12 de junho de 2003) foi um
professor americano de bioestatística na Escola Mailman de Saúde Pública
da Universidade de Columbia , onde também atuou como chefe da Divisão de
Bioestatística de 1975 a 1992. Ele é conhecido. por seu trabalho em
estatísticas de saúde mental , particularmente avaliando a
confiabilidade das classificações diagnósticas e as medidas, modelos e
controle de erros na categorização.

Fleiss (1981) oferece uma classificação Kappa que pode nos ajudar a
interpretar os coeficientes obtidos:

\begin{itemize}
\tightlist
\item
  Entre 0,40 e 0,60: concordância Regular
\item
  Entre 0,61 e 0,75: concordância Boa 8 Acima de 0,75: concordância
  Excelente
\end{itemize}

Altman (1991) propõe uma classificação um tanto mais ampla:

\begin{itemize}
\tightlist
\item
  De 0 a 0,20: concordância muito fraca
\item
  De 0,21 a 0,40: concordância fraca
\item
  De 0,41 a 0,60: concordância moderada
\item
  De 0.61 a 0,80: concordância boa
\item
  De 0.81 a 1,00: concordância muito boa
\end{itemize}

A função R \textbf{kappam.fleiss()} do \textbf{pacote irr} pode ser
usada para calcular Fleiss kappa como um índice de concordância entre
avaliadores entre m avaliadores em dados categóricos.

\bookmarksetup{startatroot}

\hypertarget{summary}{%
\chapter{Summary}\label{summary}}

\bookmarksetup{startatroot}

\hypertarget{referuxeancias}{%
\chapter*{Referências}\label{referuxeancias}}
\addcontentsline{toc}{chapter}{Referências}

\markboth{Referências}{Referências}

COHEN J. A (1960). Coefficient of agreement for nominal scales. Journal
of Educational and Measurement, v.20, n.1, p.37-46.

FLEISS, J.L. (1981). Statistical methods for rates and proportions. New
York: John Wiley and Sons.

ALTMAN, D.G. (1991). Practical statistics for medical research. New
York: Chapman and Hall.

Conover, W. J. Practical nonparametric statistics. 2a. ed.~New York:
John Wiley \& Sons, 1999.

Lehmann, E.L.; D'Abrera, H.J.M. Nonparametrics: Statistical Methods
Based on Ranks. Holden-Daym, California, 1975. p.~264.

Siegel, S., Castellan, Jr., N. J., Estatística não-paramétrica para
ciências do comportamento. São Paulo: Bookman (Artmed), 2006.

Agresti, A -- An Introduction to Categorical Data Analysis (Wiley Series
in Probability and Statistics) -- 2ª edição. New York, USA: Wiley, 2007.

Triola, M. F. Introdução à estatística. Rio de Janeiro: Livros Técnicos
e Científicos, 2013.



\end{document}
